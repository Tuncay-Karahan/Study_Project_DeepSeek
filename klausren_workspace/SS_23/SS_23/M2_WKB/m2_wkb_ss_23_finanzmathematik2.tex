\begin{Aufgabe}[3] 
Ehepaar Schmidt hat vor 5 Jahren 10.000 Euro für seine Tochter Lina angelegt, in der Hoffnung 20 Jahre später (also in 15 Jahren) 20.000 Euro angespart zu haben. In den ersten Jahren gab es zweimal 1\%, zweimal 2\% und einmal 3\% Zinsen pro Jahr.
\begin{enumerate}
	\item Bestimmen Sie den effektiven Jahreszins für die ersten 5 Jahre (auf 1 Nachkommastelle gerundet).
	\item Angenommen der Zinssatz p für die kommenden 15 Jahre ist fest. Welchen Wert muss p (mindestens) haben, damit 20 Jahre nach Anlegen der 10.000 Euro tatsächlich (mindestens) 20.000 Euro angespart sind? Runden Sie auf 2 Nachkommastellen.
\end{enumerate}
	
\end{Aufgabe}

\Loesung{}{
\begin{enumerate}[label={\alph*)}] 
	\item \[p^*=100\cdot(\sqrt[5]{1.01^2\cdot 1.02^2\cdot 1.03}-1)\approx1.8\] \hfill\Punkte{1}
	\item Es gilt \hfill\Punkte{2}
	\[10000\cdot 1.01^2\cdot 1.02^2\cdot 1.03\cdot (1+\frac{p}{100})^{15}=20000\]
	Umstellen nach p ergibt
	\[p\approx 4.11\]
\end{enumerate}
}

\newpage

\endinput