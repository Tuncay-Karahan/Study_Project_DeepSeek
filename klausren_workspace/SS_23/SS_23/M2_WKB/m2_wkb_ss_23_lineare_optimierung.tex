\begin{Aufgabe}[9]

Gegeben ist das lineare Optimierungsproblem 
\[ 
 f(\vec x)=\vec c\cdot \vec x\stackrel{!}{=}\mathrm{Max}, \quad \mathbf A\vec x\leq \vec b,\quad \vec x\geq 0, 
\] 
mit 
\[ 
	\vec    c = \left(\begin{array}{c}  2\\ 1 \end{array}\right),\quad 
	\mathbf A = \left(\begin{array}{cc} -1 & 1\\ 0 & 1\\ 1 & 0 \end{array}\right),\quad 
	\vec    b = \left(\begin{array}{c}  0\\ 2\\ 4 \end{array}\right)\,. 
\] 

\begin{enumerate}
\item 	Zeichnen Sie den zulässigen Bereich in das gegebene Koordinatensystem ein.
\item 	Zeichnen Sie im zulässigen Bereich alle Punkte $(x_1,x_2)$ mit 
		$f(x_1,x_2)=6$ ein. 
\item 	Wenden Sie den Primalen Simplex-Algorithmus auf das lineare Optimierungsproblem an.
%\item	Wäre das gegebene Optimierungsproblem für 
%		$\vec c=\left(\begin{array}{c} 1\\ 1 \end{array}\right)$ lösbar? 
%		Begründen Sie Ihre Antwort. 
%\item 	Wäre das gegebene Optimierungsproblem für 
%		$\vec b=\left(\begin{array}{c} -6\\ 2 \\ 2 \end{array}\right)$ lösbar? 
%		Begründen Sie Ihre Antwort. \\
\end{enumerate}

\begin{center} 
\begin{tikzpicture}[xscale=1,yscale=1] 
 \draw[help lines,step=0.5] (-5.5,-1.0) grid (12.0,8.0); 
 \draw[->, darkgray, line width=1pt] (-1.0,0) -- (8.0,0) node[below] {$x_1$}; 
 \draw[->, darkgray, line width=1pt] (0,-1.0) -- (0,7) node[left] {$x_2$}; 
 %\foreach \i in {-1,...,-1} { \draw (\i,0.1) -- (\i,-0.1) node[below] {$\i$}; } 
 \foreach \i in {1,...,7} { \draw (\i,0.1) -- (\i,-0.1) node[below] {$\i$}; } 
 \foreach \i in {1,...,6} { \draw (0.1,\i) -- (-0.1,\i) node[left] {$\i$}; } 
 %\foreach \i in {-1,...,-1} {\draw (0.1,\i) -- (-0.1,\i) node[left] {$\i$};} 
 \ifLoesung 
	%\coordinate[] (P) at (0,3); \fill (P) circle (2pt); 
	\coordinate[label=] (Q) at (4,0); \fill (Q) circle (2pt); 
	\coordinate[label=] (R) at (0,0); \fill (R) circle (2pt); 
	\coordinate[label= ] (S) at (2,2); \fill (S) circle (2pt); 
	\coordinate[label= ] (T) at (4,2); \fill (T) circle (2pt); 
	\coordinate[label= right: Zulässigkeitsbereich  ] (S1) at (0.5,0.5);  
	\coordinate[label= right: Isolinie ] (S2) at (1.75,1.5); 

	\draw[ultra thick,-] (0,0)--(2,2); 
	\draw[ultra thick,-] (2,2)--(4,2); 
	\draw[ultra thick,-] (4,2)--(4,0);
	\draw[ultra thick,-] (0,0)--(4,0); 
	%\draw[ultra thick,-] (0,0)--(0,3); 
	\draw[gray, dashed, ultra thick,-] (2,2)--(3,0); 
\fi
 \end{tikzpicture} 
\end{center} 

\newpage


\Loesung{}{

% ----------------------------------------------------------------------------
% MATLAB-Quelltext
% p  = 5;
% K0 = 30000;
% N1 = -K0 + 15000/(1+p/100) + 17000/(1+p/100)^2
% N2 = -K0 + 16500/(1+p/100) + 16000/(1+p/100)^2
% K2 = K0*(1+5/(12*100))^(12*2)
% pstern= 100*((1+p/(12*100))^12-1)
% ----------------------------------------------------------------------------

\begin{enumerate}
\item Zulässigkeitsbereich: siehe Skizze. 				
	\hfill\Punkte{2} 

\item Isolinie $f(x_1,x_2)=6$: siehe Skizze.			
	\hfill\Punkte{1}

\item 
	Das LOP ist in Standardformat gegeben und wegen $\vec 0 \leq \vec b$ ist $\vec 0$ ein Eckpunkt des zulässigen Bereichs. \\ 
	Das Ausgangstableau lautet 
	\[ 
	\begin{array}{c|cc|ccc|c} 
	\mbox{BV} &  x_1                   & x_2                    & z_1                    & z_2                    & z_3                    & b_i  \\ \hline 
	z_1      & -1 & 1 & 1 & 0 & 0 & 0  \\ 
	z_2      & 0 & 1 & 0 & 1 & 0 & 2 \\ 
	z_3      & 1 & 0 & 0 & 0 & 1 & 4 \\ \hline 
	f        & 2 & 1 & 0 & 0 & 0 & 0 \\ 
	\end{array} 
	\hfill\Punkte{1}
	\] 
$1$. Schritt: 
	Pivot-Spalte: $1$; Pivot-Zeile: $3$. Die Nichbasisvariable $x_{1}$ wird in die Basis aufgenommen. Dies führt auf das Tableau 
	\[ 
	\begin{array}{c|cc|ccc|c} 
	\mbox{BV}   &  x_1                    & x_2                     & z_1                     & z_2                     & z_3                     & b_i  \\ \hline 
	z_1 & 0 & 1 & 1 & 0 & 1 & 4\\ 
	z_2 & 0 & 1 & 0 & 1 & 0 & 2\\ 
	x_{1}  & 1 & 0 & 0 & 0 & 1 & 4\\ \hline 
	f          & 0 & 1 & 0 & 0 & -2 & -8\\ 
	\end{array} 
	\hfill\Punkte{2}
	\] 
$2$. Schritt: 
	Pivot-Spalte: $2$; Pivot-Zeile: $2$. 
	Die Nichbasisvariable $x_{2}$ wird in die Basis aufgenommen. Dies führt auf das Tableau 
	\[ 
	\begin{array}{c|cc|ccc|c} 
	\mbox{BV}   &  x_1                    & x_2                     & z_1                     & z_2                     & z_3                     & b_i  \\ \hline 
	z_{1}  & 0 & 0 & 1 & -1 & 1 & 2\\ 
	x_2 & 0 & 1 & 0 & 1 & 0 & 2\\ 
	x_{1}  & 1 & 0 & 0 & 0 & 1 & 4\\ \hline 
	f          & 0 & 0 & 0 & -1 & -2 & -10\\ 
	\end{array} 
	\hfill\Punkte{2}
	\] 
Die Zielfunktionszeile enthält nur Elemente $\leq 0$. Dies entspricht der 1. Abbruchbedingung. \\ 
Die optimale zulässige Lösung ist $(x_1,x_2)=(4,2)$ mit $f(x_1,x_2)= 10$. 
%Bemerkung: Falls die letzte Zeile bei Gau\ss-Operationen skaliert wurde, muss diese Skalierung am Ende r\"uckg\"angig gemacht werden, damit der Wert in der Ecke unten rechts stimmt. 
\hfill\Punkte{1}
\end{enumerate}

}

\ifLoesung
\else
\newpage
\Loesung{}{}
\fi

\end{Aufgabe}

\newpage

\endinput