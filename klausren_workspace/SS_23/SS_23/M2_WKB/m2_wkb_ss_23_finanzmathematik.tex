\begin{Aufgabe}[6] 
	Frau R. wird in 10 Jahren in Rente gehen und denkt über verschiedene Möglichkeiten des Geldanlegens nach, um ihre Rente aufzubessern. In den folgenden Aufgaben wird von einem jährlichen Zinssatz von 3\% ausgegangen.
	\begin{enumerate}
		\item Frau R. denkt zunächst daran, heute genügend Geld anzulegen, um mit Rentenbeginn bis zu 30 Jahre lang (also 30 mal) jeweils zu Beginn des Rentenjahres 6000 Euro abheben zu können. Welche Summe müsste Frau R. dafür heute mindestens anlegen?
		\item Frau R. fällt auf, dass sie die Inflationsrate einbeziehen sollte - sie geht von jährlich 3\% aus. Sie plant also mit ihrem Renteneintritt in zehn Jahren 6000 Euro und zu Beginn jedes weiteren Rentenjahres 3\% mehr als im Vorjahr abheben zu können - wieder bis zu 30 Jahre lang.
		\begin{itemize}
			\item Welche Summe plant Frau R. zu Beginn ihres 2., 5. bzw. 30. Rentenjahres abheben zu können?
			\item Welche Summe müsste zum Renteneintritt in 10 Jahren angespart sein, damit 30 Jahre lang wie beschrieben jeweils zu Beginn des Rentenjahres Geld abgehoben werden kann?
			\item Welche Summe müsste heute angelegt werden?
		\end{itemize}
	
	\end{enumerate}
	





\Loesung{}{
	\begin{enumerate}[label={\alph*)}]
		\item Der Endwert nach 30 Jahren Rente liegt bei \hfill\Punkte{3}
		\[R_{30}=E\cdot q\cdot\frac{q^{30}-1}{q-1}=6000\cdot 1.03\cdot\frac{1.03^{30}-1}{0.03}\approx 294016.07 \quad \text{(Euro)}\]
		Dieser Wert muss um 40 Jahre zurückdiskontiert werden und man erhält
		\[K_0=\frac{294016.07}{1.03^{40}}\approx90132.64 \quad \text{(Euro)}\]
		Hierbei wurde aufgerundet!
		\item Für das 2., 5. bzw. 30 Rentenjahr sind folgende Abhebungen geplant: \hfill\Punkte{1}
		\[6000\cdot 1.03=6180, \qquad 6000\cdot 1.03^{4}\approx 6753.05,\qquad 6000\cdot 1.03^{29}\approx 14139.39 \quad \text{(Euro)}\]
		Die angesparte Summe zum Renteneintritt muss bei \hfill\Punkte{1}
		\[6000+\frac{6000\cdot 1.03}{1.03}+\frac{6000\cdot 1.03^2}{1.03^2}+...+\frac{6000\cdot 1.03^{29}}{1.03^{29}}=30\cdot 6000=180000 \]
		Euro liegen. \\
		Diese Summe um 10 Jahre zurückdiskontiert, erhält man die Summe, die heute anzulegen ist. \hfill\Punkte{1}
		\[K_0=\frac{180000}{1.03^{10}}\approx 133936.91 \quad \text{(Euro)}\]
		Hier wurde wieder aufgerundet!
	\end{enumerate}
}

	\ifLoesung
\else
\newpage
\Loesung{}{}
\fi

\end{Aufgabe}

\newpage

\endinput