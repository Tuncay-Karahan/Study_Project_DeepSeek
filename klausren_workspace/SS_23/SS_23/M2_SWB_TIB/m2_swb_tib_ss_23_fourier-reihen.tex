\begin{Aufgabe}[10]
	Gegeben ist die periodische Funktion $f$, mit
	\[
	f(t) = \e^{t/2} \mbox{ für } t \in [-\pi, \pi), \quad f(t + 2 \, \pi) = f(t) \, .
	\]
	\begin{enumerate}
		\item
			Skizzieren Sie die Funktion $f$ für $t \in [-\pi, 4 \, \pi]$.
		\item
			Bestimmen Sie eine Formel für die komplexen Fourier-Koeffizienten $c_k$ der Funktion $f$, für $k = 0,1,2,3,\ldots$
		\item
			Welchen Mittelwert $m$ hat die Funktion $f$?
		\item
			An welchen Stellen tritt bei der Funktion $f$ das Gibbsche Phänomen auf?
			Wie groß sind die Überschwinger?
	\end{enumerate}
	
	\begin{tikzpicture}[xscale=1.0,yscale=1.0]
		\draw[line width=0.25pt,step=0.5cm,color=lightgray] (-4.0,-6.0) grid(13.5cm,6.0cm);
		\draw[->,thick] (-3.5, 0.0) -- (13.0, 0.0) node[below=1mm] {$t$};
		\draw[->,thick] ( 0.0,-5.5) -- ( 0.0,5.5) node[left]  {$f$};
		\foreach \i in {-5,-4,-3,-2,-1,1,2,3,4,5} {
			\draw[thick] (0.1,\i) -- (-0.1,\i) node[left] {$\i$};
		}
		\draw[thick] (-3.14,0.1) -- (-3.14,-0.1) node[below] {$\vphantom{2}-\pi$};
		\draw[thick] (  3.14,0.1) -- (  3.14,-0.1) node[below] {$\vphantom{2}\pi$};
		\draw[thick] (  6.28,0.1) -- (  6.28,-0.1) node[below] {$2 \, \pi$};
		\draw[thick] (  9.42,0.1) -- (  9.42,-0.1) node[below] {$3 \, \pi$};
		\draw[thick] (12.56,0.1) -- ( 12.56,-0.1) node[below] {$4 \, \pi$};
		\ifLoesung
		\draw[ultra thick, smooth, domain=-3.14:3.14,color=red] plot (\x,{exp(0.5*\x)}) ;
		\draw[ultra thick, smooth, domain=3.14:9.42,color=red] plot (\x,{exp(0.5*(\x-6.28))}) ;
		\draw[ultra thick, smooth, domain=9.42:12.56,color=red] plot (\x,{exp(0.5*(\x-12.56))}) ;
		\fi
	\end{tikzpicture}
	
	\Loesung{}{
		\begin{enumerate}[series=fourierreihen_loesung]
			\item
			Skizze:
			\hfill\Punkte{2 P}
		\end{enumerate}
	}
	
	\newpage
	
	\Loesung{}{
		\begin{enumerate}[resume=fourierreihen_loesung]
			\item
			Formel:
			\hfill\Punkte{1 P}
			\[
			c_k
			= \frac{1}{T} \int_{-T/2}^{T/2} f(t) \, \e^{- \mathrm{i} \, k \, \omega \, t}  \mathrm{d} \, t
			= \frac{1}{2 \, \pi} \int_{-\pi}^{\pi} \e^{t/2}  \e^{- \mathrm{i} \, k \, t}  \mathrm{d} \, t \, .
			\]
			Stammfunktion:
			\hfill\Punkte{2 P}
			\[
			c_k
			= \frac{1}{2 \, \pi} \int_{-\pi}^{\pi} \e^{t/2 - \mathrm{i} \, k \, t}  \mathrm{d} \, t
			= \frac{1}{2 \, \pi} \int_{-\pi}^{\pi} \e^{\left(1/2 - \mathrm{i} \, k \right) t}  \mathrm{d} \, t
			= \frac{1}{2 \, \pi} \left[ \frac{ \e^{\left(1/2 - \mathrm{i} \, k \right) t} }{1/2 - \mathrm{i} \, k } \right]_{- \, \pi}^\pi \, .
			\]
			Grenzen einsetzen und vereinfachen:
			\hfill\Punkte{2 P}
			\[
			c_k
			= \frac{ \e^{\left(1/2 - \mathrm{i} \, k \right) \pi} -  \e^{\left(1/2 - \mathrm{i} \, k \right) (- \pi)}}{2 \, \pi \left(1/2 - \mathrm{i} \, k \right)}
			= \frac{ \e^{\pi / 2- \mathrm{i} \, k \, \pi} - \e^{- \pi / 2 + \mathrm{i} \, k \, \pi}}{\pi - \mathrm{i} \, k \, 2 \, \pi}
			= \frac{ (-1)^k(\e^{\pi / 2}  - \e^{-\pi / 2} )}{\pi - \mathrm{i} \, k \, 2 \, \pi} \, .
			\]
			\item
			Mittelwert
			\hfill\Punkte{1 P}
			\[
			m
			= c_0
			=  \frac{\e^{\pi / 2}  - \e^{-\pi / 2}}{\pi} \approx 1.465 \, .
			\]
			\item
			Gibbsche Phänomen:
			\hfill\Punkte{1 P}
			\[
			t_k = \pi + k \cdot 2 \, \pi, \quad k = 0,1,2,3, \ldots \, .
			\]
			Überschwinger:
			\hfill\Punkte{1 P}
			\[
			\pm 0.09 \cdot (\e^{\pi / 2} - \e^{-\pi / 2}) \approx \pm 0. 41 \, .
			\]
		\end{enumerate}
	}
	
\end{Aufgabe}

\newpage

\endinput