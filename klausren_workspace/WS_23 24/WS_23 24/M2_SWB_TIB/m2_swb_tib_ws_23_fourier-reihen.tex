\begin{Aufgabe}[8]
	Gegeben ist die periodische Funktion $f$, mit
	\[
		f(t) = t^2 \mbox{ für } t \in [-2, 2), \quad f(t + 4) = f(t) \, .
	\]
	\begin{enumerate}
		\item
			Skizzieren Sie die Funktion $f$ für $t \in [-2, 10]$.
		\item
			Bestimmen Sie den Mittelwert $m$ der Funktion $f$.
		\item
			An welchen Stellen tritt bei der Funktion $f$ das Gibbsche Phänomen auf?
		\item
			Im folgenden bezeichnen $a_k$, $b_k$ die reellen und $c_k$ die komplexen Fourier-Koeffizienten der Funktion $f$.
			Welche der folgenden Aussagen ist wahr und welche ist falsch?
			
			\ifLoesung 
			\begin{tabular}{p{0.3\textwidth}p{0.2\textwidth}p{0.2\textwidth}}
				Alle $a_k$ sind null                & $\square$ wahr               & {\textcolor{red}X} falsch\\
				Alle $b_k$ sind null                & {\textcolor{red}X} wahr  &              $\square$ falsch\\
				Alle $c_k$ sind reell               &  {\textcolor{red}X} wahr &              $\square$ falsch\\
				Alle $c_k$ sind rein imaginär & $\square$ wahr              & {\textcolor{red}X} falsch\\
			\end{tabular}
			\hfill\Punkte{2 P}
			\else
			\begin{tabular}{p{0.3\textwidth}p{0.2\textwidth}p{0.2\textwidth}}
				Alle $a_k$ sind null                 & $\square$ wahr & $\square$ falsch\\
				Alle $b_k$ sind null                 & $\square$ wahr & $\square$ falsch\\
				Alle $c_k$ sind reell                & $\square$ wahr & $\square$ falsch\\
				Alle $c_k$ sind rein imaginär & $\square$ wahr & $\square$ falsch\\
			\end{tabular}
			\fi
	\end{enumerate}
	
	\begin{tikzpicture}[xscale=1.0,yscale=1.0]
		\draw[line width=0.25pt,step=0.5cm,color=lightgray] (-4.0,-6.0) grid(13.5cm,6.0cm);
		\draw[->,thick] (-3.5, 0.0) -- (13.0, 0.0) node[below=1mm] {$t$};
		\draw[->,thick] ( 0.0,-5.5) -- ( 0.0,5.5) node[left]  {$f$};
		\foreach \i in {-5,-4,-3,-2,-1,1,2,3,4,5} {
			\draw[thick] (0.1,\i) -- (-0.1,\i) node[left] {$\i$};
		}
		\foreach \i in {-3,-2,-1,1,2,3,4,5,6,7,8,9,10,11,12} {
			\draw[thick] (\i,0.1) -- (\i,-0.1) node[below] {$\i$};
		}
		
		\ifLoesung
		\draw[ultra thick, smooth, domain=-2:2,color=red] plot (\x,{\x*\x});
		\draw[ultra thick, smooth, domain=2:6,color=red] plot (\x,{(\x-4)*(\x-4)});
		\draw[ultra thick, smooth, domain=6:10,color=red] plot (\x,{(\x-8)*(\x-8)}) ;
		\fi
	\end{tikzpicture}
	
	\Loesung{}{
		\begin{enumerate}[series=fourierreihen_loesung]
			\item
				Skizze:
				\hfill\Punkte{2 P}
		\end{enumerate}
	}
	
	\newpage
	
	\Loesung{}{
		\begin{enumerate}[resume=fourierreihen_loesung]
			\item
				Mittelwert:
				\hfill\Punkte{3 P}
				\[
				m
				= \frac{1}{T} \int_{-T/2}^{T/2} f(t) \, \mathrm{d} \, t
				= \frac{1}{4} \int_{-2}^{2} t^2 \, \mathrm{d} \, t
				= \frac{1}{4} \left[ \frac{t^3}{3} \right]_{-2}^{2} 
				= \frac{16}{12}
				= \frac{4}{3} \, .
				\]
				\item
				Die Funktion $f$ ist stetig, deshalb tritt das Gibbsche Phänomen nicht auf.
				\hfill\Punkte{1 P}
		\end{enumerate}
	}
	
\end{Aufgabe}

\newpage

\endinput