\begin{Aufgabe}[10]% Siehe SS 16
	Bestimmen Sie die allgemeine reelle Lösung des Differenzialgleichungssystems
	\[
	\begin{array}{ccrcrcr}
		\dot{x}_1 & = & -2 \, x_1 & + & 3 \, x_2 & + & 6 \, \e^{-2 \, t} \, ,\\
		\dot{x}_2 & = &  2 \, x_1 &  + & 3 \, x_2 & .   &  \\
	\end{array}
	\]
	
	\Loesung{}{
		Matrixform:
		\hfill\Punkte{1 P}
		\[
		\dot{\mathbf{x}}
		=
		\left(
		\begin{array}{rr}
			-2 & 3\\
			2 &  3\\
		\end{array}
		\right)
		\mathbf{x} +
		\left(
		\begin{array}{c}
			 6 \, \e^{-2 \, t}\\
			0\\
		\end{array}
		\right)
		\, .
		\]
		Eigenwerte:
		\hfill\Punkte{2 P}
		\[
		\left|
		\begin{array}{cc}
			-2-\lambda &            3\\
			2                & 3 - \lambda\\
		\end{array}
		\right|
		=(-2-\lambda)(3-\lambda)-3 \cdot 2 = \lambda^2-\lambda-12 = 0
		\, \Longrightarrow \,
		\lambda_{1,2} = \frac{1\pm\sqrt{1+48}}{2} = 4 | -3 \, .
		\]
		Eigenvektor $\mathbf{v}_1$ zu $\lambda_1=4$:
		\hfill\Punkte{1 P}
		\[
		\left(
		\begin{array}{cc}
			-2-4 &                 3\\
			2                     & 3-4\\
		\end{array}
		\right)
		\mathbf{v}_1
		=
		\left(
		\begin{array}{rr}
			-6 &  3\\
		 2  & -1\\
		\end{array}
		\right)
		\mathbf{v}_1
		= \mathbf{0}
		\quad \Longrightarrow \quad
		\mathbf{v}_1
		=
		\left(
		\begin{array}{c}
			1\\
			2\\
		\end{array}
		\right) \, .
		\]
		Eigenvektor $\mathbf{v}_2$ zu $\lambda_2=-3$:
		\hfill\Punkte{1 P}
		\[
		\left(
		\begin{array}{cc}
			-2+3 &                 3\\
			2                     & 3+3\\
		\end{array}
		\right)
		\mathbf{v}_2
		=
		\left(
		\begin{array}{cc}
			1 &  3\\
			2  & 6\\
		\end{array}
		\right)
		\mathbf{v}_2
		= \mathbf{0}
		\quad \Longrightarrow \quad
		\mathbf{v}_2
		=
		\left(
		\begin{array}{r}
			3\\
			-1\\
		\end{array}
		\right) \, .
		\]
		Allgemeine reelle Lösung des homogenen Differenzialgleichungssystems:
		\hfill\Punkte{1 P}
		\[
		\mathbf{x}_h
		= C_1 \, \e^{4 \, t}\
		\left(
		\begin{array}{c}
			1\\
			2\\
		\end{array}
		\right)
		+ C_2 \, \e^{- 3 \, t}
		\left(
		\begin{array}{r}
		 3\\
			-1\\
		\end{array}
		\right),
		\quad C_1, C_2\in \mathbb{R} \, .
		\]
	}
	
	\newpage
	
	\Loesung{}{
		Ansatz für partikuläre Lösung ohne Resonanz:
		\hfill\Punkte{1 P}
		\[
			\mathbf{x}_p
			= \e^{- 2 \, t}\
			\left(
			\begin{array}{c}
				A\\
				B\\
			\end{array}
			\right) \, .
		\]
		Ableitung in Differenzialgleichungssystem eingesetzt:
		\hfill\Punkte{1 P}
		\[
			-2 \, \e^{- 2 \, t}\
			\left(
			\begin{array}{c}
				A\\
				B\\
			\end{array}
			\right)
			=
			\left(
			\begin{array}{rr}
				-2 & 3\\
				2 &  3\\
			\end{array}
			\right)
			\e^{- 2 \, t}\
			\left(
			\begin{array}{c}
				A\\
				B\\
			\end{array}
			\right) +
			\left(
			\begin{array}{c}
				6 \, \e^{-2 \, t}\\
				0\\
			\end{array}
			\right)
			\, .
		\]
		Lineares Gleichungssystem:
		\hfill\Punkte{1 P}
		\[
			\begin{array}{rcrcrcr}
				-2 \, A & = & -2 \, A & + & 3 \, B & + & 6 \\
				-2 \, B & = &   2 \, A & + & 3 \, B &    & \\
			\end{array}
			\quad \Longrightarrow \quad
			B = -2, \, A = 5 \, .
		\]
		Allgemeine reelle Lösung des Differenzialgleichungssystems:
		\hfill\Punkte{1 P}
		\[
			\mathbf{x} = \mathbf{x}_h + \mathbf{x}_p
			= C_1 \, \e^{4 \, t}\
			\left(
			\begin{array}{c}
				1\\
				2\\
			\end{array}
			\right)
			+ C_2 \, \e^{- 3 \, t}
			\left(
			\begin{array}{r}
				3\\
				-1\\
			\end{array}
			\right) +
			e^{- 2 \, t}
			\left(
			\begin{array}{r}
				 5\\
				-2\\
			\end{array}
			\right),
			\quad C_1, C_2\in \mathbb{R} \, .
		\]
	}
	
\end{Aufgabe}

\newpage

\endinput