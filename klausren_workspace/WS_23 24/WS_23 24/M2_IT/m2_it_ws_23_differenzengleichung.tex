\begin{Aufgabe}[8] 
	Eine Differenzengleichung erster Ordnung ist gegeben durch
	\[
		x_{k+1} = 1.05 \, x_k - 1, \quad x_0 = 10, \quad k = 0,1,2,3,\ldots . 
	\]
	\begin{enumerate}
		\item
			Geben Sie die Zahlenwerte von $x_1$ und $x_2$ an.
		\item
		  	Bestimmen Sie die Lösung der Differenzengleichung.
		 \item
		 	Für welche Indizes $k$ ist $x_k < 0$?  
	\end{enumerate}
	
	\Loesung{}{
		\begin{enumerate}
			\item
				Zahlenwerte:
				\hfill\Punkte{1 P}
				\[
					x_0 = 1.05 \cdot 10 - 1 = 9.5, \quad x_1 = 1.05 \cdot 9.5 - 1 = 8.975
				\]
			\item
				Lösungsformel:
				\hfill\Punkte{1 P}
				\[
					x_{k+1} = \lambda \, x_k + r_k
					\quad \Longrightarrow \quad
					x_k = \lambda^k \cdot x_0 + \sum_{l=0}^{k-1} \lambda^{k-1-l} r_l \, .
				\]
				$\lambda = 1.05$, $r_k = -1$, $x_0 = 1$:
				\hfill\Punkte{1 P}
				\[
					x_k = 1.05^k \cdot 10 + \sum_{l=0}^{k-1}1.05^{k-1-l} \cdot (-1) \, .
				\]
				Geometrische Reihe:
				\hfill\Punkte{1 P}
				\[
					x_k =
					1.05^k \cdot 10 - \frac{1 - 1.05^k}{1 - 1.05} \, .
				\]
				Vereinfachung:
				\hfill\Punkte{1 P}
				\[
					x_k =
					10 \cdot 1.05^k  - \frac{1 - 1.05^k}{- 0.05} =
					10 \cdot 1.05^k  + 20 - 20 \cdot 1.05^k =
					- 10 \cdot 1.05^k  + 20\, .
				\]
			\item
				Bedingung $x_k < 0$:
				\hfill\Punkte{1 P}
				\[
					- 10 \cdot 1.05^k  + 20 < 0
					\quad \Longleftrightarrow \quad
					20 < 10 \cdot 1.05^k
					\quad \Longleftrightarrow \quad
					2 < 1.05^k \, .
				\]
				Nach $k$ auflösen:
				\hfill\Punkte{1 P}
				\[
					\ln(2) < \ln\left((1.05)^k\right)
					\quad \Longleftrightarrow \quad
					\ln(2) < k \, \ln(1.05)
					\quad \Longleftrightarrow \quad
					k > \frac{\ln(2)}{\ln(1.05)} \approx 14.2
				\]
				Ab $k = 15$ sind alle Folgenglieder negativ.
				\hfill\Punkte{1 P}
		\end{enumerate}
	}
	
	\newpage
	
	\Loesung{}{
		Alternative Lösung für \textbf{b)}
		\par
		\vspace*{1cm}
		\par
		Homogene Lösung:
		\[
			x_{k+1} -1.05 \, x_k = 0
			\quad \Longrightarrow \quad
			\lambda - 1.05 = 0
			\quad \Longrightarrow \quad
			\lambda = 1.05
			\quad \Longrightarrow \quad
			x_k^h = C \cdot 1.05^k \, .
		\]
		Partikuläre Lösung
		\[
			x_k^p = A
			\quad \Longrightarrow \quad
			A - 1.05 \, A = - 1
			\quad \Longrightarrow \quad
			A ( 1 - 1.05 ) = -1
			\quad \Longrightarrow \quad
			A = \frac{-1}{-0.05} = 20 \, .
		\]
		Allgemeine Lösung:
		\[
			x_k = x_k^h + x_k^p = C \cdot 1.05^k + 20 \, .
		\]
		Anfangswert $x_0 = 10$: 
		\[
			k = 0: \quad 
			x_0 = C \cdot 1.05^0 + 20 
			\quad \Longrightarrow \quad
			10 = C + 20
			\quad \Longrightarrow \quad
			C = -10 \, .
		\]
		Ergebnis:
		\[
			x_k = -10 \cdot 1.05^k + 20 \, .
		\]
	}
	
\end{Aufgabe} 

\newpage

\endinput