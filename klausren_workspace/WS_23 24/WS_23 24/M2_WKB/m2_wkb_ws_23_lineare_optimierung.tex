\begin{Aufgabe}[9]

Gegeben ist das lineare Optimierungsproblem 
\[ 
f(\vec x)=\vec c\cdot \vec x\stackrel{!}{=}\mathrm{Max}, \quad \mathbf A\vec x\leq \vec b,\quad \vec x\geq 0, 
\] 
mit 
\[ 
\vec    c = \left(\begin{array}{c}  -1\\ 2 \end{array}\right),\quad 
\mathbf A = \left(\begin{array}{cc} -1 & 1\\ 1 & 1\\ 1 & 0 \end{array}\right),\quad 
\vec    b = \left(\begin{array}{c}  1\\ 5\\ 3 \end{array}\right)\,. 
\] 

\begin{enumerate}
	\item 	Zeichnen Sie den zulässigen Bereich in das gegebene Koordinatensystem ein.
	\item 	Zeichnen Sie im zulässigen Bereich alle Punkte $(x_1,x_2)$ mit 
	$f(x_1,x_2)=1$ ein. 
	\item 	Wenden Sie den Primalen Simplex-Algorithmus auf das lineare Optimierungsproblem an.
\end{enumerate}

\begin{center} 
	\begin{tikzpicture}[xscale=1,yscale=1] 
		\draw[help lines,step=0.5] (-5.5,-1.0) grid (12.0,9.0); 
		\draw[->, darkgray, line width=1pt] (-1.5,0) -- (5.5,0) node[below] {$x_1$}; 
		\draw[->, darkgray, line width=1pt] (0,-1.0) -- (0,6.5) node[left] {$x_2$}; 
		\foreach \i in {-1,...,-1} { \draw (\i,0.1) -- (\i,-0.1) node[below] {$\i$}; } 
		\foreach \i in {1,...,5} { \draw (\i,0.1) -- (\i,-0.1) node[below] {$\i$}; } 
		\foreach \i in {1,...,6} { \draw (0.1,\i) -- (-0.1,\i) node[left] {$\i$}; } 
		\foreach \i in {-1,...,-1} {\draw (0.1,\i) -- (-0.1,\i) node[left] {$\i$};} 
		\ifLoesung 
		\coordinate[] (P) at (0,1); \fill (P) circle (2pt); 
		\coordinate[label=] (Q) at (3,0); \fill (Q) circle (2pt); 
		\coordinate[label=] (R) at (0,0); \fill (R) circle (2pt); 
		\coordinate[label= ] (S) at (2,3); \fill (S) circle (2pt); 
		\coordinate[label= ] (T) at (3,2); \fill (T) circle (2pt); 
		\coordinate[label= right: Zulässigkeitsbereich  ] (S1) at (0.0,0.25);  
		\coordinate[label= right: Isolinie ] (S2) at (2,1.5); 
		
		\draw[ultra thick,-] (0,1)--(2,3); 
		\draw[ultra thick,-] (2,3)--(3,2); 
		\draw[ultra thick,-] (3,0)--(3,2);
		\draw[ultra thick,-] (0,0)--(3,0); 
		\draw[ultra thick,-] (0,0)--(0,1); 
		\draw[gray, dashed, ultra thick,-] (0,0.5)--(3,2); 
		\fi
	\end{tikzpicture} 
\end{center} 

\clearpage
\Loesung{}{
	
	% ----------------------------------------------------------------------------
	% MATLAB-Quelltext
	% p  = 5;
	% K0 = 30000;
	% N1 = -K0 + 15000/(1+p/100) + 17000/(1+p/100)^2
	% N2 = -K0 + 16500/(1+p/100) + 16000/(1+p/100)^2
	% K2 = K0*(1+5/(12*100))^(12*2)
	% pstern= 100*((1+p/(12*100))^12-1)
	% ----------------------------------------------------------------------------
	
	\begin{enumerate}
		\item Zulässigkeitsbereich: siehe Skizze. 				
		\hfill\Punkte{2} 
		
		\item Isolinie $f(x_1,x_2)=1$: siehe Skizze.			
		\hfill\Punkte{1}
		
		\item 
		Das LOP ist in Standardformat gegeben und wegen $\vec 0 \leq \vec b$ ist $\vec 0$ ein Eckpunkt des zulässigen Bereichs. \\ 
		Das Ausgangstableau lautet 
		\[ 
		\begin{array}{c|cc|ccc|c} 
			\mbox{BV} &  x_1                   & x_2                    & z_1                    & z_2                    & z_3                    & b_i  \\ \hline 
			z_1      & -1 & 1 & 1 & 0 & 0 & 1  \\ 
			z_2      & 1 & 1 & 0 & 1 & 0 & 5 \\ 
			z_3      & 1 & 0 & 0 & 0 & 1 & 3 \\ \hline 
			f        & -1 & 2 & 0 & 0 & 0 & 0 \\ 
		\end{array} 
		\hfill\Punkte{1}
		\] 
		$1$. Schritt: 
		Pivot-Spalte: $2$; Pivot-Zeile: $1$. Die Nichbasisvariable $x_{2}$ wird in die Basis aufgenommen. Dies führt auf das Tableau 
		\[ 
		\begin{array}{c|cc|ccc|c} 
			\mbox{BV}   &  x_1                    & x_2                     & z_1                     & z_2                     & z_3                     & b_i  \\ \hline 
			x_2 & -1 & 1 & 1 & 0 & 0 & 1\\ 
			z_2 & 2 & 0 & -1 & 1 & 0 & 4\\ 
			z_3  & 1 & 0 & 0 & 0 & 1 & 3\\ \hline 
			f          & 1 & 0 & -2 & 0 & 0 & -2\\ 
		\end{array} 
		\hfill\Punkte{2}
		\] 
		$2$. Schritt: 
		Pivot-Spalte: $1$; Pivot-Zeile: $2$. 
		Die Nichbasisvariable $x_{1}$ wird in die Basis aufgenommen. Dies führt auf das Tableau 
		\[ 
		\begin{array}{c|cc|ccc|c} 
			\mbox{BV}   &  x_1                    & x_2                     & z_1                     & z_2                     & z_3                     & b_i  \\ \hline 
			x_2  & 0 & 1 & \frac{1}{2} & \frac{1}{2} & 0 & 3\\ 
			x_1 & 1 & 0 & -\frac{1}{2} & \frac{1}{2} & 0 & 2\\ 
			z_3  & 0 & 0 & \frac{1}{2} & -\frac{1}{2} & 1 & 1\\ \hline 
			f          & 0 & 0 & -\frac{3}{2} & -\frac{1}{2} & 0 & -4\\ 
		\end{array} 
		\hfill\Punkte{2}
		\] 
		Die Zielfunktionszeile enthält nur Elemente $\leq 0$. Dies entspricht der 1. Abbruchbedingung. \\ 
		Die optimale zulässige Lösung ist $(x_1,x_2)=(2,3)$ mit $f(x_1,x_2)= 4$. 
		\hfill\Punkte{1}
		
		
		
		
		
	\end{enumerate}
	%\smallskip
%	\HFehler{
%		\begin{itemize}
			
%		\end{itemize}
%	}
	
	
}



\ifLoesung
\else
\newpage
\Loesung{}{}
\fi



\end{Aufgabe}

\newpage

\endinput