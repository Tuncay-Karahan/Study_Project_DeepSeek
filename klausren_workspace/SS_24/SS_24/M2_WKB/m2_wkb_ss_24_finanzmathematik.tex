\begin{Aufgabe}[8] 
Insa möchte ihrem Bruder Till \textbf{5000 Euro} leihen, die ihm noch fehlen, um ein Auto finanzieren zu können. In dieser Aufgabe werde mit einem Zinssatz von \textbf{2\% p.a.} gerechnet.
\begin{enumerate}
	\item 
	Insa möchte, dass ihr Bruder in \textbf{5 jährlichen Raten} von jeweils \textbf{1000 Euro} den Kredit zurückzahlt. Dabei soll nach einem Jahr die erste Rate gezahlt werden.
	Welchen Verlust macht Insa bei dieser Rückzahlunsvariante, wenn als Zeitpunkt für die Berechnung die Auszahlung der 5000 Euro an Till genommen wird?
	\item 
	Durch welche jährliche Rate E müssten die 1000 Euro in Aufgabe a) ersetzt werden, damit Insa keinen Verlust macht?
	\item
	Till ist seiner Schwester besonders dankbar und macht daher folgenden Vorschlag: Er könnte unbegrenzt jedes Jahr eine Rate von \textbf{400 Euro} zahlen ohne jemals die Zahlungen einzustellen. Auch bei dieser Variante werde die erste Rate nach einem Jahr bezahlt. Welchen Barwert hätte diese Zahlungsfolge, wenn man annimmt, dass die Zahlungen tatsächlich niemals aufhören? 
	
\end{enumerate}

\Loesung{}{
\begin{enumerate}
	\item Der Barwert der Zahlungsfolge liegt bei 
	\hfill\Punkte{2 P}
	\[B=\frac{1000}{1.02}+\frac{1000}{1.02^2}+\frac{1000}{1.02^3}+\frac{1000}{1.02^4}+\frac{1000}{1.02^5}=1000\cdot \frac{1.02^5-1}{1.02^5\cdot (1.02-1)}\approx4713.46\;\text{(Euro)}\]
	Insas Verlust bezogen auf den Auszahlungszeitpunkt beträgt damit 
	\hfill\Punkte{1 P}
	\[5000-4713,46=286.54 \;\text{(Euro)}\]
	\item 
	Setze
	\hfill\Punkte{2 P}
	\[5000=E\cdot \frac{1.02^5-1}{1.02^5\cdot (1.02-1)}\]
	Umstellen nach E ergibt
	\[E=5000\cdot \frac{1.02^5\cdot (1.02-1)}{1.02^5-1}\approx1060.79 \; \text{(Euro)}\]
	\item 
	Für den Barwert B der unendlichen Zahlungsfolge ergibt sich mit der geometrischen Reihe 
	\hfill\Punkte{3 P}
	\[B=400\cdot(\frac{1}{1.02}+\frac{1}{1.02^2}+\frac{1}{1.02^3}+...)=400\cdot\frac{1}{1.02}\cdot \frac{1}{1-\frac{1}{1.02}}=400\cdot \frac{1}{1.02}\cdot \frac{1.02}{1.02-1}\]
	und damit 
	\[B=\frac{400}{0.02}=20000 \; \text{(Euro)}\]
\end{enumerate}
}
 \ifLoesung
\else
\newpage
\Loesung{}{}
\fi

\end{Aufgabe}

\newpage


\endinput