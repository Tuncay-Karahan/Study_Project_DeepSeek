\begin{Aufgabe}[9]
	
Gegeben ist das lineare Optimierungsproblem 
\[ 
f(\vec x)=\vec c\cdot \vec x\stackrel{!}{=}\mathrm{Max}, \quad \mathbf A\vec x\leq \vec b,\quad \vec x\geq 0, 
\] 
mit 
\[ 
\vec    c = \left(\begin{array}{c} 2\\ 6 \end{array}\right),\quad 
\mathbf A = \left(\begin{array}{cc} 2 &-2\\ -1 & 2 \end{array}\right),\quad 
\vec    b = \left(\begin{array}{c} 0\\ 4 \end{array}\right)\,. 
\] 
\begin{enumerate} 
	\item Zeichnen Sie den zulässigen Bereich in das gegebene Koordinatensystem ein. 
	\item Zeichnen Sie im zulässigen Bereich alle Punkte $(x_1,x_2)$ mit $f(x_1,x_2)=12$ ein. 
	\item Wenden Sie den Primalen Simplex-Algorithmus auf dieses Optimierungsproblem an. 
	\item Wäre das gegebene Optimierungsproblem für \[\mathbf A = \left(\begin{array}{cc} -2 & 2\\ -1 & 2 \end{array}\right) \quad  \] lösbar? 
	Begründen Sie Ihre Antwort. 
	 
	 
\end{enumerate} 
	
	\begin{center}  
		\begin{tikzpicture}[xscale=1,yscale=1]  
			\draw[help lines,step=0.5] (-2.5,-5.0) grid (15.0,8.5);  
			\draw[->, darkgray, line width=1pt] (-1.5,0) -- (8.0,0) node[below] {$x_1$};  
			\draw[->, darkgray, line width=1pt] (0,-1.5) -- (0,7.0) node[left] {$x_2$};  
			\foreach \i in {-1,...,-1} { \draw (\i,0.1) -- (\i,-0.1) node[below] {$\i$}; }  
			\foreach \i in {1,...,7} { \draw (\i,0.1) -- (\i,-0.1) node[below] {$\i$}; }  
			\foreach \i in {1,...,6} { \draw (0.1,\i) -- (-0.1,\i) node[left] {$\i$}; }  
			\foreach \i in {-1,...,-1} {\draw (0.1,\i) -- (-0.1,\i) node[left] {$\i$};}  
			\ifLoesung  
			\coordinate[] (P) at (0,2); \fill (P) circle (2pt);  
			% \coordinate[label=] (Q) at (0,2); \fill (Q) circle (2pt);  
			\coordinate[label=] (R) at (0,0); \fill (R) circle (2pt);  
			% \coordinate[label= ] (S) at (2,4); \fill (S) circle (2pt);  
			\coordinate[label= ] (T) at (4,4); \fill (T) circle (2pt);  
			
			% Dreieck zeichnen -------------------------------------------------
			\draw[fill=black!25,thick] 
			(0,2) node (A) [label=right:$$] {} --
			(4,4) node (B) [label=above:$$] {} --
			(0,0) node (C) [label=above left:$$] {} -- 
			cycle;
			% ------------------------------------------------------------------
			\coordinate[label= right: Zulässigkeitsbereich  ] (S1) at (0.75,2.75);   
			\coordinate[label= right: Isolinie ] (S2) at (1.5,1.25);  
			
			%\draw[ultra thick,-] (2,0)--(4,4);  
			% \draw[ultra thick,-] (2,4)--(4,2);  
			%\draw[ultra thick,-] (0,0)--(4,4); 
			% \draw[ultra thick,-] (0,0)--(2,0);  
			%\draw[ultra thick,-] (0,0)--(0,2);  
			\draw[gray, dashed, ultra thick,-] (0, 2)--(1.5, 1.5);  
			\fi  
		\end{tikzpicture}  
	\end{center}  
	\clearpage 
	
	
	\Loesung{}{
		
	\begin{enumerate} 
		\item Zulässigkeitsbereich: siehe Skizze.     \hfill\Punkte{2} 
		\item Isolinie $f(x_1,x_2)=6$: siehe Skizze.  \hfill\Punkte{1} 
		\item Das LOP ist in Standardformat gegeben, d.h. eine zulässige Basislösung existiert. \\ 
		Das Ausgangstableau lautet 
		\[ 
		\begin{array}{c|cc|cc|c} 
			\mbox{BV} &  x_1 & x_2 & z_1 & z_2 & b_i  \\ \hline 
			z_1      & 2 & -2 & 1 & 0 & 0  \\ 
			z_2      & -1 & 2 & 0 & 1 & 4  \\ \hline 
			f        & 2 & 6 & 0 & 0 & 0  \\ 
		\end{array} 
		\hfill\Punkte{1} 
		\] 
		$1$. Schritt: 
		Pivot-Spalte: $2$; Pivot-Zeile: $2$. Die Nichbasisvariable $x_{2}$ wird in die Basis aufgenommen. Dies führt auf das Tableau 
		\[ 
		\begin{array}{c|cc|cc|c} 
			\mbox{BV}   &  x_1                    & x_2                     & z_1                     & z_2                     & b_i  \\ \hline 
			z_1 & 1 & 0 & 1 & 1 & 4 \\ 
			x_{2}  & -\frac{1}{2} & 1 & 0 & \frac{1}{2} & 2 \\ \hline 
			f          & 5 & 0 & 0 & -3 & -12 \\ 
		\end{array} 
		\hfill\Punkte{2} 
		\] 
		$2$. Schritt: 
		Pivot-Spalte: $1$; Pivot-Zeile: $1$. Die Nichbasisvariable $x_{1}$ wird in die Basis aufgenommen. Dies führt auf das Tableau 
		\[ 
		\begin{array}{c|cc|cc|c} 
			\mbox{BV}   &  x_1                    & x_2                     & z_1                     & z_2                     & b_i  \\ \hline 
			x_{1}  & 1 & 0 & 1 & 1 & 4 \\ 
			x_{2}  & 0 & 1 & \frac{1}{2} & 1 & 4 \\ \hline 
			f          & 0 & 0 & -5 & -8 & -32 \\ 
		\end{array} 
		\hfill\Punkte{2} 
		\] 
		Die Zielfunktionszeile enthält nur Elemente $\leq 0$. Dies entspricht der 1. Abbruchbedingung. \\ Die optimale zulässige Lösung ist $(x_1,x_2)=(4,4)$ mit $f(x_1,x_2)= 32$. \\ 
		 
		\item Nein, denn der Zulässigkeitsbereich wäre unbeschränkt. \hfill\Punkte{1} 
	\end{enumerate} 	
	}
	
	\ifLoesung
	\else
	\newpage
	\Loesung{}{}
	\fi
	
\end{Aufgabe}

\newpage

\endinput