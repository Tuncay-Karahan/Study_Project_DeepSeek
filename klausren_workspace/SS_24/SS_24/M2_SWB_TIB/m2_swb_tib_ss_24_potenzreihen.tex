\begin{Aufgabe}[8]% vgl. WS 17/18
	Gegeben ist die Funktion $f$ mit 
	\[
		f(x)=\e^x \, \ln(1+x) \, .
	\]
	\begin{enumerate}
		\item
			Geben Sie die Potenzreihen der Funktionen $g$ und $h$ an der Entwicklungstelle $x_0 = 0$ an, mit
			\[
				g(x)=\e^x, \quad h(x) =\ln(1+x) \, .
			\]
			Welche Konvergenzradien haben diese Potenzreihen?
		\item
			Bestimmen Sie das Taylor-Polynom $T_3$ von Grad $3$ der Funktion $f$ an der Entwicklungstelle $x_0 = 0$ mithilfe der Potenzreihen der Funktionen $g$ und $h$ aus Aufgabenteil \textbf{a)}.
		\item
			Berechnen Sie einen Näherungswert $\tilde{I}$ für das bestimmte Integral
			\[
				I = \int_0^1 f(x) \, \mbox{d} \, x \approx \tilde{I} = \int_0^1 T_3(x) \, \mbox{d} \, x
			\]
			mithilfe des Taylor-Polynoms $T_3$ aus Aufgabenteil \textbf{b)}.
	\end{enumerate}
	%
	%
	%
	\Loesung{}{
		\begin{enumerate}[series=potenzreihen_loesung]
			\item
				Potenzreihe der $\e$-Funktion:
				\hfill\Punkte{1 P}
				\[
					\e^x = \sum_{k=0}^{\infty}\frac{1}{k!}x^k = 1 + x + \frac{x^2}{2} + \frac{x^3}{6} \pm \ldots
				\]
				Potenzreihe des Logarithmus:
				\hfill\Punkte{1 P}
				\[
					\ln(1 + x) = \sum_{k=1}^{\infty}\frac{(-1)^{k-1}}{k}x^k = x-\frac{x^2}{2}+\frac{x^3}{3} \pm \ldots
				\]
				Die Potenzreihe der $\e$-Funktion hat den Konvergenzradius $\infty$ und die Potenzreihe der $\ln$-Funktion hat den Konvergenzradius $1$. 
				\hfill\Punkte{1 P}
			\item
				Cauchy-Produkt:
				\hfill\Punkte{1 P}
				\[
					f(x) = g(x) \cdot h(x) = \left( 1 + x + \frac{x^2}{2} + \frac{x^3}{6} \pm \ldots \right) \left( x-\frac{x^2}{2}+\frac{x^3}{3}  \pm \ldots \right)
				\]
				Taylor-Polynom $T_3$ von Grad $3$:
				\hfill\Punkte{2 P}
				\[
					T_3(x) = 1 \cdot \left( x-\frac{x^2}{2}+\frac{x^3}{3} \right) + x \cdot \left( x-\frac{x^2}{2} \right) + \frac{x^2}{2} \cdot x = x + \frac{x^2}{2} + \frac{x^3}{3} 
				\]
			\item
				Näherungswert:
				\hfill\Punkte{2 P}
				\[
					\tilde{I} = \int_0^1 T_3(x) \, \mbox{d} \, x =  \int_0^1 x + \frac{x^2}{2} + \frac{x^3}{3} \, \mbox{d} \, x 
					= \left[ \frac{x^2}{2} + \frac{x^3}{6} + \frac{x^4}{12} \right]_0^1
					= \frac{1}{2} + \frac{1}{6} + \frac{1}{12} = \frac{9}{12}  = \frac{3}{4} 
				\]
		\end{enumerate}
	}
	
	\newpage

		\ifLoesung
		\else
		\Loesung{}{}
		\fi
		
\end{Aufgabe}

\newpage

\endinput