\begin{Aufgabe}[10]% vgl. SS 18
	Wir betrachten das Anfangswertproblem
	\[
		y'(x) \cdot y(x) = \sin(x), \quad y(0) = -1 \, .
	\]
	\begin{enumerate}
		\item
			Handelt es sich um eine lineare oder um eine nichtlineare Differenzialgleichung?
		\item
			Bestimmen Sie die exakte Lösung des Anfangswertproblems.
		\item
			Berechnen Sie einen Näherungswert für $y(1)$, indem Sie mit der Schrittweite $h=\displaystyle\frac{1}{2}$ zwei Schritte mit dem Polygonzugverfahren von Euler durchführen.
		\item
			Wie groß ist die Abweichung des in Aufgabenteil \textbf{c)} berechneten Näherungswerts von der exakten Lösung?
	\end{enumerate}
	
	\Loesung{}{
		\begin{enumerate}[series=dgl_erster_ordnung]
			\item
			Nichtlineare Differenzialgleichung.
			\hfill\Punkte{1 P}
			\item		
			Separation:
			\hfill\Punkte{1 P}
			\[
				\frac{\mbox{d} \, y}{\mbox{d} \, x} \cdot y = \sin(x)
				\quad \Longrightarrow \quad
				\int y \, \mbox{d} \, y = \int \sin(x) \, \mbox{d} \, x
			\]
			Integration:
			\hfill\Punkte{1 P}
			\[
				\frac{1}{2} \, y^2 = -\cos(x) + C
			\]
			Allgemeine Lösung:
			\hfill\Punkte{1 P}
			\[
				y(x) = \pm \sqrt{2 \, C - 2 \, \cos(x)}
			\]
			Anfangswert $y(0) = -1$:
			\hfill\Punkte{1 P}
			\[
				\frac{1}{2} \cdot (-1)^2 = -\cos(0) + C
				\quad \Longrightarrow \quad
				C = \frac{3}{2}
			\]
			Lösung des Anfangswertproblems:
			\hfill\Punkte{1 P}
			\[
				y(x) = -\sqrt{3 - 2 \, \cos(x)}
			\]
			\item
			1. Schritt mit $x_0=0$ und $y_0 = -1$:
			\hfill\Punkte{1 P}
			\[
				y_1 = y_0 + h \, \frac{\sin(x_0)}{y_0} = -1, \quad x_1 = x_0 + h = \frac{1}{2}
			\]
			2. Schritt:
			\hfill\Punkte{1 P}
			\[
				y_2 = y_1 + h \, \frac{\sin(x_1)}{y_1} = -1 - \frac{1}{2} \sin\left( \frac{1}{2} \right) \approx -1.2397,  \quad x_2 = x_1 + h = 1
			\]
		\end{enumerate}
	}
	
	\newpage
	
	\Loesung{}{
		\begin{enumerate}[resume=dgl_erster_ordnung]
			\item
			Exakte Lösung:
			\hfill\Punkte{1 P}
			\[
				y(1) = -\sqrt{3 - 2 \, \cos(1)} \approx - 1.3854
			\]
			Abweichung:
			\hfill\Punkte{1 P}
			\[
				\left| y(1) - y_2 \right| \approx \left| - 1.3854 + 1.2397 \right| \approx 0.1457
			\]
		\end{enumerate}	
		}
\end{Aufgabe}

\newpage

\endinput