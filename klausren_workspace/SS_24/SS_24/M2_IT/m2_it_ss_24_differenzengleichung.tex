%
% Differenzengleichung ähnlich zu WS 23/24, da Ergebnisse der Studierenden im WS 23/24 nicht gut waren!
% 
\begin{Aufgabe}[10] 
	Eine Differenzengleichung erster Ordnung ist gegeben durch
	\[
		20 \, x_{k+1} - 21 \, x_k = -200, \quad x_0 = 100, \quad k = 0,1,2,3,\ldots . 
	\]
	\begin{enumerate}
		\item
			Geben Sie die Zahlenwerte von $x_1$ und $x_2$ an.
		\item
			Bestimmen Sie die Lösung der Differenzengleichung.
		\item
			Interpretieren Sie die Zahlenfolge $(x_k)$ als Kontostand nach $k$ Jahren, eines Kontos mit festem Zinsatz von dem am Ende jeden Jahres der selbe Betrag abgehoben wird.
			Wie hoch sind Zinsatz und Betrag?
			Wie oft kann der Betrag von dem Konto abgehoben werden bevor der Kontostand negativ wird?
	\end{enumerate}
	
	\Loesung{}{
		\begin{enumerate}[series=differenzengleichung]
			\item
				Rekursionsformel auflösen:
				\hfill\Punkte{1 P}
				\[
					x_{k+1} = \frac{21}{20} \, x_k - 10 \, .
				\]
				Zahlenwerte:
				\hfill\Punkte{1 P}
				\begin{flalign*}
					& k = 0: \, x_1 = \frac{21}{20}\, x_0 - 10  = \frac{21}{20} \cdot 100 - 10 = 95 \, .\\
					& k = 1:  \, x_2 = \frac{21}{20}\, x_1 - 10  = \frac{21}{20} \cdot  95 - 10 = 89.75 \, .
				\end{flalign*}
			\item
				Homogene Lösung:
				\hfill\Punkte{1 P}
				\[
					20 \, x_{k+1} - 21 \, x_k = 0,
					\quad \Longrightarrow \quad
					20 \, \lambda - 21 = 0
					\quad \Longrightarrow \quad
					\lambda = \frac{21}{20}
					\quad \Longrightarrow \quad
					x_k^h = C \cdot \left( \frac{21}{20} \right)^k \, .
				\]
				Partikuläre Lösung:
				\hfill\Punkte{1 P}
				\[
					x_k^p = A
					\quad \Longrightarrow \quad
					20 \, A - 21 \, A = - 200
					\quad \Longrightarrow \quad
					A = 200 \, .
				\]
				Allgemeine Lösung:
				\hfill\Punkte{1 P}
				\[
					x_k = x_k^h + x_k^p = C \cdot \left( \frac{21}{20} \right)^k + 200 \, .
				\]
				Anfangswert $x_0 = 100$: 
				\hfill\Punkte{1 P}
				\[
					k = 0: \quad 
					x_0 = C \cdot \left( \frac{21}{20} \right)^0 + 200
					\quad \Longrightarrow \quad
					100 = C + 200
					\quad \Longrightarrow \quad
					C = -100 \, .
				\]
				Ergebnis:
				\hfill\Punkte{1 P}
				\[
					x_k = -100 \cdot \left( \frac{21}{20} \right)^k + 200 \, .
				\]
		\end{enumerate}
	}
	
	\newpage
	
	\Loesung{}{
		\begin{enumerate}[resume=differenzengleichung]
			\item
				Zinssatz $\frac{21}{20} - 1 = \frac{1}{20} = 5 \%$, Betrag $10$.
				\hfill\Punkte{1 P}
				\par
				Bedingung $x_k = 0$:
				\hfill\Punkte{1 P}
				\[
					100 \cdot \left( \frac{21}{20} \right)^k = 200
					\quad \Longleftrightarrow \quad
					\left( \frac{21}{20} \right)^k  = 2
				\]
				Nach $k$ auflösen:
				\hfill\Punkte{1 P}
				\[
					\ln(2) = \ln\left( \frac{21}{20} \right)^k
					\quad \Longleftrightarrow \quad
					\ln(2) = k \, \ln\left( \frac{21}{20} \right)
					\quad \Longleftrightarrow \quad
					k = \frac{\ln(2)}{ \ln\left( \frac{21}{20} \right)} \approx 14.2
				\]
				Der Betrag kann $14$ mal abgehoben werden. 
		\end{enumerate}
		\par
		\vspace*{10mm}
		\par
		Alternative Lösung für \textbf{b)} mit Lösungsformel für lineare Differenzengleichung erster Ordnung:
		\[
			x_{k+1} = \lambda \, x_k + r_k
			\quad \Longrightarrow \quad
			x_k = \lambda^k \cdot x_0 + \sum_{l=0}^{k-1} \lambda^{k-1-l} r_l \, .
		\]
		Mit $\lambda = \frac{21}{20}$, $r_k = -10$, $x_0 = 100$:
		\[
			x_k =  \left( \frac{21}{20} \right)^k \cdot 100 + \sum_{l=0}^{k-1}\left( \frac{21}{20} \right)^{k-1-l} \cdot (-10) \, .
		\]
		Geometrische Reihe:
		\[
			\sum_{l=0}^{k-1}q^{k-1-l}  = \frac{1- q^k}{1 - q}\, .
		\]
		Mit $q = \frac{21}{20} $:
		\[
			x_k
			= 100 \, \left( \frac{21}{20} \right)^k - 10 \, \frac{1 - \left( \frac{21}{20} \right)^k}{1 - \frac{21}{20}} 
			= 100 \, \left( \frac{21}{20} \right)^k + 200 \, \left(1 - \left( \frac{21}{20} \right)^k \right)
			= -100 \cdot \left( \frac{21}{20} \right)^k + 200 \, .
		\]
	}
	
\end{Aufgabe} 

\newpage

\endinput